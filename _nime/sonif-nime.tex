% Template for NIME 2017
%
% Modified by Cumhur Erkut on <2016-10-11 Tue>
% Modified by Edgar Berdahl on 5 November 2014
% Modified by Baptiste Caramiaux on 25 November 2013
% Modified by Kyogu Lee on 7 October 2012
% Modified by Georg Essl on 7 November 2011
%
% Based on "sig-alternate.tex" V1.9 April 2009
% This file should be compiled with "nime-alternate.cls"


\documentclass{nime-alternate}
%\usepackage{dirtytalk}

\begin{document}
%
% --- Author Metadata here ---
\conferenceinfo{NIME'17,}{May 15-19, 2017, Aalborg University Copenhagen, Denmark.}

\title{??? Paper Title ???}

%
% You need the command \numberofauthors to handle the 'placement
% and alignment' of the authors beneath the title.
%
% For aesthetic reasons, we recommend 'three authors at a time'
% i.e. three 'name/affiliation blocks' be placed beneath the title.
%
% NOTE: You are NOT restricted in how many 'rows' of
% "name/affiliations" may appear. We just ask that you restrict
% the number of 'columns' to three.
%
% Because of the available 'opening page real-estate'
% we ask you to refrain from putting more than six authors
% (two rows with three columns) beneath the article title.
% More than six makes the first-page appear very cluttered indeed.
%
% Use the \alignauthor commands to handle the names
% and affiliations for an 'aesthetic maximum' of six authors.
% Add names, affiliations, addresses for
% the seventh etc. author(s) as the argument for the
% \additionalauthors command.
% These 'additional authors' will be output/set for you
% without further effort on your part as the last section in
% the body of your article BEFORE References or any Appendices.

\numberofauthors{6}
%
\author{
% You can go ahead and credit any number of authors here,
% e.g. one 'row of three' or two rows (consisting of one row of three
% and a second row of one, two or three).
%
% The command \alignauthor (no curly braces needed) should
% precede each author name, affiliation/snail-mail address and
% e-mail address. Additionally, tag each line of
% affiliation/address with \affaddr, and tag the
% e-mail address with \email.
%
% 1st. author
\alignauthor
Ben Trovato\\
       \affaddr{Institute for Clarity in Documentation}\\
       \affaddr{1932 Wallamaloo Lane}\\
       \affaddr{Wallamaloo, New Zealand}\\
       \email{trovato@corporation.com}
% 2nd. author
\alignauthor
G.K.M. Tobin\\
       \affaddr{Institute for Clarity in Documentation}\\
       \affaddr{P.O. Box 1212}\\
       \affaddr{Dublin, Ohio 43017-6221}\\
       \email{webmaster@marysville-ohio.com}
% 3rd. author
\alignauthor Lars Th{\o}rv{\"a}ld\\
       \affaddr{The Th{\o}rv{\"a}ld Group}\\
       \affaddr{1 Th{\o}rv{\"a}ld Circle}\\
       \affaddr{Hekla, Iceland}\\
       \email{larst@affiliation.org}
\and  % use '\and' if you need 'another row' of author names
% 4th. author
\alignauthor Lawrence P. Leipuner\\
       \affaddr{Brookhaven Laboratories}\\
       \affaddr{Brookhaven National Lab}\\
       \affaddr{P.O. Box 5000}\\
       \email{lleipuner@researchlabs.org}
% 5th. author
\alignauthor Sean Fogarty\\
       \affaddr{NASA Ames Research Center}\\
       \affaddr{Moffett Field}\\
       \affaddr{California 94035}\\
       \email{fogartys@amesres.org}
% 6th. author
\alignauthor Charles Palmer\\
       \affaddr{Palmer Research Laboratories}\\
       \affaddr{8600 Datapoint Drive}\\
       \affaddr{San Antonio, Texas 78229}\\
       \email{cpalmer@prl.com}
}
% There's nothing stopping you putting the seventh, eighth, etc.
% author on the opening page (as the 'third row') but we ask,
% for aesthetic reasons that you place these 'additional authors'
% in the \additional authors block, viz.
%\additionalauthors{Additional authors: John Smith (The Th{\o}rv{\"a}ld Group,
%email: {\texttt{jsmith@affiliation.org}}) and Julius P.~Kumquat
%(K. Consortium, email: {\texttt{jpkumquat@consortium.net}}).}
\date{20 Jan 2017}
% Just remember to make sure that the TOTAL number of authors
% is the number that will appear on the first page PLUS the
% number that will appear in the \additionalauthors section.

\maketitle
\begin{abstract}

The abstract should preferably be between 100 and 200 words.
\end{abstract}

\keywords{sonification, ???}


\acmclassification1998{

H.5.5 [Information Interfaces and Presentation] Sound and Music Computing, H.5.2 [Information Interfaces and Presentation] User Interfaces---Haptic I/O, I.2.9 [Artificial Intelligence] Robotics---Propelling mechanisms. ?\textbf{?? TO DO}}



\section{Introduction}

- motivation

- challenges

\textbf{- the Vicon system}\\ \par
With advanced technology and intelligent controls, Vicon Vantage represents a powerful tool to capture motion.Vicon Vantage system used in this project contains 8's 5 megapixel camera which transmit movement in realtime to Nexus or Blade. Having accelerometers and temperature sensors, each camera detects any decalibration of the system. \footnote{See  \url{https://www.vicon.com}.}.

\section{State of the Art}

- Vicon \& related projects

- interactive / movement sonification examples\cite{hermann2011sonification}.

- \textit{Sound in space} represents another innovative element in this project because introduces the ideea of \textit{controling sound} by movement.
%GRIG: trebuie o referinta aici. Noi inventam acest concept? Avem nevoie de un concept nou?
This means that the sound is an entity, gets a materialization and it becomes switchable \cite{soundunseen}. It is not about the \textbf{localization} of sound in space (detecting the direction of sound source), it is about its \textbf{position} in space, the coordinates of the sound object in space, like an actual \textit{object} in a room. \textit{Sound objects} concept represents an innovative tool for multimedia arts such as sketches, imaginary games and realtime interactions.
%GRIG: As incerca sa reformulam paragraful asta; distinctia intre localization si position este prea vaga.

%GRIG: mutat de la 3.2
Interaction between sound control and human gesture has constantly increased over the last years \cite{Gestureanalysis}. Probabilistic models for analysing motion and sound relationships became a necessity and a forthcoming tool \cite{probabilisticmodels}.

\section{Project Description}

\subsection{Concept}

- Performance aesthetic\\

\textbf{- Gestures, virtual objects, dynamic mapping}\\ \par
Manny interactions and programable elements which are included in the concept have an exactly function and usefulness. These elements are created in MAX with the help of Vicon SDK and OSC C++ library. The concept suppose an interactive action of a performer or a simple user with an virtual object which has associated gestures defined by person. Virtual object interaction acts in sound design utility like a master controller. Searching for a certain object comes with an audio feedback which makes the search easier. Object's asscociated gestures are compatible with sound design patch and contribute at audio performance. A dynamic mapping of marker's coordinates is necessary to transfer data between Nexus and MAX.

- Visual environment

\subsection{Implementation}
\textbf{- Character design (Nexus)}\\ \par
Implementation of the concept presented above requires Nexus, Vicon SDK and MAX sotware. Every character involved in the scene is defined by a limited number of markers. In this case, two markers are positioned on the head, one marker is positioned on elbow and the others 2 on the hand (thumb and index finger). Every marker has associated a name in Nexus and between them 6 segments are drawn. It is very important in realtime capture motion, that the marker to have assigned correct coordinates.

- \textbf{Vicon extensions (SDK plugin)} \par
Vicon's SDK is a versatile and simple tool for users to gain easy access to Vicon DataFlow created in Nexus, Blade or Tracker applications.The Vicon DataStream Software Development Kit (SDK) provides intuitive programable access to data with custom functions created in C++. With the help of some functions, Vicon's SDK forwardes the Vicon DataStream to other constructive softwares and plug-ins to create custom applications \footnote{See \url{https://www.vicon.com/products/software/datastream-sdk/}.}. In combination with Open Sound Control protocol, Vicon's SDK forwards data to any software compatible with this communication protocol (eg. MAX). OSC is a protocol for communication among computers, sound synthesizers and other multimedia devices\footnote{See  \url{http://opensoundcontrol.org}.}. Hence, any marker can be routed in MAX using its parameters and coordinates. Also, the Vicon DataStream Software Development Kit (SDK) admits inside changes such as labeling markers, timecode generation and framerate.

\subsubsection{Max modules}
\textbf{- Objects generation \& performance mechanics}\\ \par
%GRIG: nu e cazul sa prezentam Max pe larg aici. Eventual o propozitie scurta mai sus...
%Max is a realtime visual programing environment for music and multimedia arts that helps you build stand-alone applications, plugins and mixing audio signals. In order to create interactive sounds, attractive grapichs and special effects, MAX creates a connection between virtual objects and subpatches\footnote{See \url{http://www.cycling74.com/}.}.
Manipulating objects algorithm consists of some big steps: object generation, finding the object, picking up the object, throwing the object on the floor. 
%GRIG: ar mai fi picking up the object, si eventual throwing...

Object generation is realized by random generators with the help of \textit{drunk} object, but with certain limits. These limitations are influenced by the dimensions of the room in which the Vicon system is installed. Finding the object supposes continuous mathematic operations between the coordinates of the object and coordinates of the left hand's marker. This process comes with an audio feedback. When these coordinates are close enough one to another, the object is retreived and manipulated by performer (eg. define gesture). After all these processes, a simple comparison between the coordinates of the floor and the value of the z axes of the marker is done in order to put down the object. According to this, a performer can handle as many objects as he wants.\\ \par
\textbf{- Gesture recognition}\\ \par
 \textit{Mubu} containers provided by Ircam laboratories in MAX software represent a handy tool to record and analyze gesture, captured with Vicon system \cite{mubu}. Our gesture recognition algorithm is based on Hierarchical Hidden Markov Models (HHMM) implemented in \textit{mubu.hhmm} object of MAX/MSP. HHMMs are a generalization of HMM where each state is considered to be a self-contained probabilistic model \cite{hhmm}. The system is trained by captured data which is essentially a gesture. This process requires a predefined indicator in order to delimitate gestures from all data flow. The algorithm analyzes all input data and generates a probability of similarity between data and saved gestures. In order to control every generated object, there are associated 2 or 3 gestures saved by the performer, but there is a limited time for the gestures to be executed. Predefined gestures offer the possibility to delete the gesture just saved and also indicate the moment the gesture is recorded.

- Sound design

- Visualisation (jitter)

\section{Case studies}
\subsection{Interactive Installation}
\subsection{Performance}
- Solo / duet / tutti ...


\section{Conclusions and Future Work}
- Areas of improvement

- Eye tracking?

%ACKNOWLEDGMENTS are optional
\section{Acknowledgments}
This section is optional; it is a location for you
to acknowledge grants, funding, editing assistance and
what have you. 

%
% The following two commands are all you need in the
% initial runs of your .tex file to
% produce the bibliography for the citations in your paper.
\bibliographystyle{abbrv}
\bibliography{sonif-ref} 

% That's all folks!
\end{document}
